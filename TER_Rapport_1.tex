\documentclass{report}

% Peut etre utile plus tard
% \usepackage[latin1]{inputenc}
% \usepackage[T1]{fontenc}
% \usepackage[francais]{label}

% PAGE DE GARDE
\title{TER - Rendu 1}
\author{Emile Cadorel, Guillaume Gas, Jimmy Furet, Valentin Bouziat}
\date{20 mars 2016}

\begin{document}
\maketitle

\chapter{Etude des squelettes existants}
Afin de mieux comprendre le but de notre travail, nous avons fait des recherches sur la programmation de squelettes. Nous nous sommes ainsi intéressé à deux librairies qui semblent reprendre le concept qui nous intéresse : Thrust et SkePU.
\section{Thrust}
Cette librairie de templates C++ pour Cuda est basée sur la STL (Standard Template Library). Elle permet l'utilisation d'algorithmes parallèles avec un minimum de code. Ainsi, on retrouve les algorithmes suivant :

\begin{itemize}

\item Transformation : applique une opération sur un set de données et met le résultat dans un set déstination.
\item Reduction : utilise une opération binaire pour réduire une entrée en une unique valeur.
\item Prefix-Sums : ou opération <<scan>> permet par exemple, sur un tableau en entrée, d'obtenir en sortie un tableau avec les sommets cumulés.
\item Reordering : permet d'arranger ou partitionner des données selon des prédicats de test.
\item Sorting : différents algorithmes de tri.

\end{itemize}

\section{SkePU}
Librairie de templates C++ mettant à disposition un ensemble de squelettes génériques déstinés à faciliter la création de code à executer en parallèle. On y retrouve des fonctions similaires à Thrust.

\begin{itemize}

\item Map
\item MapReduce
\item MapOverlap
\item MapArray
\item Generate
\item Scan

\end{itemize}

\chapter{Comment intégrer ces notions dans Spoc ?}
\section{Modification de l'arbre}
blablalba

\section{Génération de code Sarek}
blabla

\end{document}
